\section*{編集後記}
雑音製作所偽述部彙報 vol.3をお届けします。って例によってもう誰も読んでいないかな。
うちのサークルのメンバーは、みんな偉くなってしまたのか、忙しいらしく、「今年の夏コミの新作は
お前のしかないからよろしく」と言われ、青森出張の新幹線の中でちまちま書いています。
前作のAM編で、次回はFMとDSPが何をやっているかを書くと予告しておきながら、DSPの部分は
書けませんでした。冬にはDSPでどういう計算をしているか書いてみようかと思っていますが
予定は未定であり決定ではないということで。あとSSBの変復調の話も書きたいな。

前回も書きましたが、IQ変調、復調は、現代の無線通信には書かせない変調、復調方式になっています。
ICチップ1個で復調も簡単にできる時代ですが、だからこそバックグラウンドにある理論を知っている
のも良いのではないでしょうか。

最後に、この同人誌はDebian/GNU Linux、\TeX Live2014、
psutils、git、GNU Makeといった、オープンソースソフトウェアを使って作成されました。
また日本語のフォントはIPAexフォントをPDFに埋め込んでいます。
このような有益なソフトウェアを開発、維持、管理していただいているすべての皆様に感謝します。
また、このページまでたどり着いてくれた読者の方(おそらくあなただけです)に感謝します。
ありがとうございました。

\begin{flushright}
2014年8月 Keisuke Nakao (@jm6xxu) 
\end{flushright}

\subsection*{参考文献}
\begin{itemize}
  \item 
    SI4825-A10データシート \texttt{http://www.silabs.com/Support Documents/TechnicalDocs/Si4825-A10.pdf}
   \end{itemize}
\clearpage
\mbox{}
\vspace{36em}\\
この作品はクリエイティブ・コモンズ・ライセンス 表示 - 継承 2.1 日本 の下に提供されています。このライセンスのコピーを見るためには、http://creativecommons.org/licenses/by-sa/2.1/jp/ をご覧ください。
