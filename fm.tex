\chapter*{DSPラジオの原理 ーFM編ー}
前作に引き続き、aitendo(\texttt{http://www.aitendo.cc}さんから発売されている
DSP AM/FMラジオキットに使われているIC SI4825-A10 のデータシートに載っている
ブロックダイアグラムをネタに、FM復調の原理を紹介する。

もちろん今回も冗長なのは百も承知で、数式の展開の過程はできるだけ省かないこととする。

\section*{FM変調}
まずFM変調波がどのような数式で書けるか考えよう。
周波数$f$の電波がアンテナに受信された時、その出力電圧を$V$とすると、
\begin{equation}
V = A\sin(2 \pi f t + \phi) = A\sin(\omega t + \phi)
\end{equation}
と書ける。ここで$A$を振幅、$\omega=2\pi f$を角振動数、$t$は時間、$\phi$を初期位相とす
る。
この$f$、すなわち$\omega$を変化させて信号を乗せるのが、FM (Frequency Modulation: 周波数変調)である。ちなみに$A$を変化させるのがAM (Amplitude Modulation: 振幅変調)、$\phi$を変化させるのがPM (Phase Modulation: 位相変調)である。
前回同様、信号を乗せる電波である搬送波(Carrier)の周波数を$f_c$、振幅を$C$とし、
\begin{equation}
V_c = C\sin(2 \pi f_c t) = C\sin\omega_c t \label{eq:FM_car}
\end{equation}
と書く。
一方、搬送波に乗せられる信号波$V_s$は、音声信号が単一の周波数$f_s$のみで構成されているとするとし、振幅をSして、以下のように書く。
\begin{equation}
V_s = S\sin(2 \pi f_s t) = S\sin\omega_s t \label{eq:FM_sig}
\end{equation}

$V_s$に応じて、$V_c$の周波数を変化させるのがFM変調であるので、変調波$V_{fm}$は
$\theta_m$を時刻$t$の関数とすると、
\begin{equation}
V_{fm} = C\sin\theta_m \label{eq:FM_mod}
\end{equation}
である。

信号入力電圧1Vあたりの周波数の変化量は、
マイク等から入力された音声信号の電圧の最大値を$S$、
その時の角周波数の変化量を$\Delta\omega$とすると、
\[\delta\omega = \frac{\Delta\omega}{S}\]
となる。$\theta_m$は
\begin{eqnarray}
\theta_m &=& \omega_ct + \delta\omega\int_{0}^{t}S\sin\omega_st dt \nonumber\\
&=& \omega_c t + \frac{\Delta\omega}{S} \int_{0}^{t}S\sin\omega_st dt \nonumber\\
&=& \omega_c t + \frac{\Delta\omega}{S}[\frac{S}{\omega_s}\cos\omega_st]_{0}^{t} \nonumber\\
&=& \omega_c t + \frac{\Delta\omega}{\omega_s}\cos\omega_st \nonumber\\
&=& \omega_c t + m\cos\omega_s t
\end{eqnarray}
となる。ここで$m$は変調指数と呼ばれる定数で、
\begin{equation}
m = \frac{\Delta\omega}{\omega_s} = \frac{\Delta f}{f_s} \label{eq:mod_factor}
\end{equation}
である。$\Delta f$は最大周波数偏位で、FM放送では75kHzと決められている。

最後に、式\ref{eq:mod_factor}を、式\ref{eq:FM_mod}に代入することで
FM変調波の一般式が求められる。
\begin{equation}
  V_m = C\sin(\omega_c t + m\cos\omega_st) \label{eq:FM}
\end{equation}

\section*{FM復調}
次は、式\ref{eq:FM}で変調された信号をIQ検波してみよう。
IQ検波については、前号を参照してほしい。

局部発信機(Local Oscillator)の周波数を$f_L$、角周波数$\omega_L = 2\pi f_L$とする。
I成分およびQ成分は、
\begin{eqnarray*}
I &=& C\sin(\omega_ct + m\cos\omega_st)\sin\omega_Lt\\
Q &=& C\sin(\omega_ct + m\cos\omega_st)\sin(\omega_Lt + \frac{\pi}{2}) \nonumber\\
&=& C\sin(\omega_ct + m\cos\omega_st)\cos(\omega_Lt)
\end{eqnarray*}
三角関数の公式
\begin{eqnarray}
\sin x \sin y &=& \frac{1}{2}{\cos(x-y) - \cos(x+y)} \nonumber\\
\sin x \cos y &=& \frac{1}{2}{\sin(x+y) + \sin(x-y)} \nonumber
\end{eqnarray}
に注意して、局部発振周波数と搬送波周波数を一致させる。すなわち$\omega_L = \omega_c$
とすると、
\begin{eqnarray}
I &=& \frac{C}{2}\{\cos(m\sin\omega_st) - \cos(2\omega_ct + m\sin\omega_st)\}\nonumber\\
Q &=& \frac{C}{2}\{\sin(m\sin\omega_st) + \sin(2\omega_ct + m\sin\omega_st)\}\nonumber
\end{eqnarray}
となり、LPF(Low Pass Filter)で、搬送波の成分を除去すると、
\begin{eqnarray}
I' &=& \frac{C}{2}\cos(m\sin\omega_st)\nonumber\\
Q' &=& \frac{C}{2}\sin(m\sin\omega_st)\nonumber
\end{eqnarray}
となる。よって
\begin{eqnarray*}
  \frac{Q'}{I'} &=& \frac{\sin(m\sin\omega_st)}{\cos(m\sin\omega_st)} = \tan(m\sin\omega_st)\\
  \therefore \sin\omega_st &=& \frac{1}{m}\tan^{-1}\frac{Q'}{I'}
\end{eqnarray*}
となり、信号波を搬送波から分離することができた。これが、FM変調波のIQ復調の原理である。
