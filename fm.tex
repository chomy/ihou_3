\chapter*{DSPラジオの原理 ーFM編ー}
前作に引き続き、aitendo(\texttt{http://www.aitendo.cc}さんから発売されている
DSP AM/FMラジオキットに使われているIC SI4825-A10 のデータシートに載っている
ブロックダイアグラムをネタに、FM復調の原理を紹介する。

もちろん今回も冗長なのは百も承知で、数式の展開の過程はできるだけ省かないこととする。

\section*{FM変調}
周波数$f$の電波がアンテナに受信された時、その出力電圧を$V$とすると、
\begin{equation}
V = A\sin(2 \pi f t + \phi) = A\sin(\omega t + \phi)
\end{equation}
と書ける。ここで$A$を振幅、$\omega=2\pi f$を角振動数、$t$は時間、$\phi$を初期位相とす
る。
この$f$、すなわち$\omega$を変化させて信号を乗せるのが、FM (Frequency Modulation: 周波数変調)である。ちなみに$A$を変化させるのがAM (Amplitude Modulation: 振幅変調)、$\phi$を変化させるのがPM (Phase Modulation: 位相変調)である。
前回同様、信号を乗せる電波である搬送波(Carrier)の周波数を$f_c$、振幅を$C$とし、
\begin{equation}
V_c = C\sin(2 \pi f_c t) = C\sin\omega_c t \label{eq:FM_car}
\end{equation}
と書く。
一方、搬送波に乗せられる信号波$V_s$は、音声信号が単一の周波数$f_s$のみで構成されているとするとし、振幅をSして、以下のように書く。
\begin{equation}
V_s = S\sin(2 \pi f_s t) = S\sin\omega_s t \label{eq:FM_sig}
\end{equation}

$V_s$に応じて、$V_c$の周波数を変化させるのがFM変調であるから、式\ref{eq:FM_car}は
\begin{equation}
V_m = C\sin(\omega_c t + m\sin\omega_st)
\end{equation}
ここで、$m$は、変調指数である。変調指数$m$は、
